% !TeX root = ../main.tex
% Add the above to each chapter to make compiling the PDF easier in some editors.

\chapter{Bandwidth Benchmark}\label{chapter:benchmark}

\section{Concept}
- Measures raw theoretical maximum bandwidth of pcie link by measuring the duration of memory copies of various chunk sizes
- Checks at which chunk sizes the bandwidth of the link is fully saturated

\section{Implementation}
- Compensates for delay - 1 packet with 4B measured as delay
- Pageable and pinned memory benchmarks measured
- Warmup-feature: first transfer usually has some sort of longer delay, compensates for that (windows-finding, verify on p6000)

\section{Results}
- Transfer durations don’t really increase until 8kb
- Due to the nature of the PCI-E packet having a max payload of 4kb
- First transfer usually has a bit longer delay (warmup?)
- On windows: not executable that calls functions, but rather nvcuda64.dll - requires compiling on windows and then using a profiling tool like AMDUprof to look at the program

\section{Discussion}

\subsection{successes}
- gives accurate reading of pcie bandwidths 
- non-linear scaling of packet transfer durations (2 packets does not equal double the duration of one packet)


\subsection{shortcomings}
- does not really compensate for other bottlenecks, as seen on time-x with gen4 link bandwidth speeds
