% !TeX root = ../main.tex
% Add the above to each chapter to make compiling the PDF easier in some editors.

\chapter{Bandwidth Benchmark}\label{chapter:benchmark}

\section{Concept}
\begin{itemize}
	\item - Measures raw theoretical maximum bandwidth of pcie link by measuring the duration of memory copies of various chunk sizes
	\item - Checks at which chunk sizes the bandwidth of the link is fully saturated
	\item - goal: insight into very basics of pcie data transfer
\end{itemize}


\section{Implementation}
\begin{itemize}
	\item - Compensates for delay / overhead - 1 packet with 4B measured as delay / overhead
	\item - Pageable and pinned memory benchmarks measured
	\item - Warmup-feature: first transfer usually has some sort of longer delay, compensates for that (windows-finding, verify on p6000)
	
	
\end{itemize}

\section{Results}
\begin{itemize}
	\item - Transfer durations don’t really increase until 8kb
	\item reason: the nature of the PCI-E packet having a max payload of 4kb
	\item - First transfer usually has a bit longer delay (warmup?)
	\item - On windows: not executable that calls functions, but rather nvcuda64.dll - requires compiling on windows and then using a profiling tool like AMDUprof to look at the program
	\item - TODO: graphs
\end{itemize}


\section{Discussion}

\subsection{successes}
\begin{itemize}
	\item - gives accurate reading of pcie bandwidths compared to theoretical maximum (gen3)
	\item - non-linear scaling of packet transfer durations (2 packets does not equal double the duration of one packet) -> overhead per transfer
\end{itemize}


\subsection{shortcomings}
\begin{itemize}
	\item - does not really compensate for other bottlenecks, as seen on time-x with gen4 link bandwidth speeds
	\item - gives little to no insight on 'what is being transferred', and more along the lines of 'how much is theoretical max bandwidth'
	\item - why warmup? not fully explained, further research needed
	\item 
\end{itemize}

